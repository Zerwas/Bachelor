\section{Witts Lemma}\label{sec:witt}
In this section we will prove Witt's Lemma and explore the structure of symplectic spaces. Witt's Lemma states that an isometry between subspaces of a finite dimensional vector space can always be extended to an isomtry on the whole space.  We will roughly follow the proof in \cite{Witt}. Since we are mainly interested in symplectic spaces we will only show Witt's Lemma for those but it also holds for unitary and orthogonal spaces.

\begin{definition}
A bilinear form $\omega$ on $V$ is called \define{symplectic} if $\omega$ is nondegenerate, i.e. $\omega(x,\ .\ )\not=0_V$ for all $x\in V\setminus\{0\}$, and $\omega(x,y)=-\omega(y,x)$ for all $x,y\in V$. Then we say $(V,\omega)$ is a \define{symplectic space}.

A finite group $G$ is called \define{symplectic} if there is a symplectic space $(V,\omega)$ such that $G\cong\Aut(V,\omega)$. 

A subspace $U\leq V$ is \define{nondegenerate} if $\omega$ restricted to $U$ is nondegenerate (iff $U\cap U^\bot=\{0\}$). 

%For $x,y\in V$ with $\omega(x,y)$ we will write $x\bot y$ and for a set $X\subseteq V$ define
%\[X^\bot=\{y\in V\mid \omega(x,y)=0\}.\]
\end{definition}
Throughout this section let $(V,\omega)$ be a finite dimensional symplectic $\F_q$ vector space.
Note that $\omega(x,y)=0$ iff $\omega(y,x)=0$, also $\omega(x,x)=0$ for all $x\in V$.
We will start of with some technical lemmas.


\FlorianSagt{make shortversion for each section?}
%\begin{lemma}
%For an $x\in V$ and a subspace $U\leq V$ with $x\notin U^\bot$ we have 
%\[\dim( x^\bot\cap U)= \dim U -1.\]
%\end{lemma}

\begin{lemma}\label{lem:dimUBot}
For a subspace $U\leq V$ we have 
\[\dim U^\bot= \dim V -\dim U.\]
\end{lemma}
\begin{proof}
Let $u_1,\dots,u_m$ be a basis of $U$ and consider the linear map
\begin{align*}g\colon V\to \F_q&&\text{ with }&&g(v)=
\begin{pmatrix}
\omega(v,u_1)\\
\vdots\\
\omega(v,u_m)
\end{pmatrix}.
\end{align*}
By definition the kernel of $g$ is $U^\bot$ and since $\omega$ is nondegenerate $g$ is also surjective. Therefore the claim follows from the Rank-Nullity Theorem.
\end{proof}


The following lemma is an immediate consequence of the previous one.
\begin{lemma}\label{lem:Ubotbot}
Let $U\leq V$. Then
\begin{enumerate}[label=\roman{*})]
\item $U$ is nondegenerate iff $V=U\oplus U^\bot$ and
\item $U^{\bot\bot}=U$.
\end{enumerate}
\end{lemma}
\begin{proof}
If $U$ is nondegenerate then $U\cap U^\bot=\{0\}$. By Lemma \ref{lem:dimUBot} \[U\oplus U^\bot=\langle U,U^\bot\rangle=V.\]
The other direction is clear from the definition

For $ii)$ note that $U\subseteq U^{\bot\bot}$. By Lemma \ref{lem:dimUBot} they also have the same dimension and are therefore equal.
\end{proof}

Next we want to better understand the structure of symplectic spaces. 
A symplectic space is a \define{hyperbolic plane} if it is 2-dimensional.


\begin{lemma}\label{lem:oneHypPlane}
There is only one hyperbolic plane up to isomorphism.
\end{lemma}
\begin{proof}
Let $(V,\omega)$, $(V',\omega')$ be hyperbolic planes with basis $r,s$ and $r',s'$, respectively. 
Then $r\mapsto r', s\mapsto \lambda s'$ with $\lambda=\omega'(r',s')^{-1}\cdot\omega(r,s)$ is an isomorphism. 
\end{proof}
Next we will strengthen this result and show that any symplectic space is the direct sum of hyperbolic planes.
%Next we will strengthen this result and show that any symplectic space is uniquely determined by its dimension.

\begin{theorem}\label{thm:hypDecomp}
Let $r\in V\setminus\{0\}$. Then there is a hyperbolic plane $U\leq V$ containing $r$ such that $V=U\oplus U^\bot$. Furthermore if $W\leq V$ with $W\bot r$ and $r\notin W$, then there is a $U$ as before that also fulfills $W\bot U$.
\end{theorem}
\begin{proof}
Since $\omega(r,r)=0$ we know that $r\in r^\bot$. Let $H\leq r^\bot$ containing $W$ such that $\langle r\rangle\oplus H=r^\bot$. Then, by Lemma \ref{lem:dimUBot}, $\dim r^\bot=\dim V-1$, $\dim H^\bot=2$, and $r\in H^\bot$. In particular $\dim V>1$.
Our goal is to show that \[V=H^\bot\oplus H,\] so $H^\bot$ would be a suitable choice for $U$.
It suffices to show that $H^\bot$ is nondegenerate, as then the claim follows from Lemma \ref{lem:Ubotbot}. Let $s\in H^\bot$ such that $H^\bot=\langle r,s\rangle$. Now $H^\bot$ is nondegenerate iff $\omega(r,s)\not=0$. 

Assume $\omega(r,s)=0$. Then $r\in s^\bot$ and $H\subseteq s^\bot$. Since, by construction, $r\notin H$ we have $r^\bot=\langle r,H\rangle=s^\bot$. Hence $r$ and $s$ are linearly dependent contradicting that $r,s$ is a basis of $H^\bot$.

\end{proof}

Using Theorem \ref{thm:hypDecomp} we can describe the structure of symplectic spaces.

\begin{corollary}\label{cor:structureSympl}
%The symplectic space $(V,\omega)$ is of even dimension $2n$ and can be written as the orthogonal sum of hyperbolic planes $U_1,\dots, U_n$, i.e. $V=U_1\oplus\dots\oplus U_n$.

Let $(V,\omega)$ be a symplectic space. Then $V$ is of even dimension $2n$ and there are hyperbolic planes $U_1,\dots, U_n\leq V$ such that $V=U_1\oplus\dots\oplus U_n$ and $U_i\bot U_j$ for $i\not=j$.
In particular for any $n$ there is exactly one symplectic space of dimension $2n$.
\end{corollary}
\begin{proof}
We use induction on $n=\dim V$.

If $n=1$ then $V=\langle v\rangle$ for $v\in V\setminus\{0\}$. But $\omega(v,v)=0$ and therefore $\omega$ is degenerate which is a contradiction.

If $n=2$ then the claim follows from Lemma \ref{lem:oneHypPlane}.

If $n>2$ then by Theorem \ref{thm:hypDecomp} there is a hyperbolic plane $U_1\leq V$ such that $V=U_1\oplus U_1^\bot$. Then, by induction hypothesis, $n$ is even and $U^\bot =U_2\oplus\dots \oplus U_m$ for some hyperbolic planes $U_2,\dots,U_m\leq U^\bot$ with $m=\frac{n}{2}$ and $U_i\bot U_j$ for $i\not=j$.
\end{proof}

%\begin{corollary}
%Every symplectic space is of even dimension and for any $n$ there is exactly one symplectic space of dimension $2n$.
%\end{corollary}
With these powerful tools we can easily prove Witt's Lemma.
%For the remainder of this section 
Let $\alpha\colon U\to W$ be an isometry between subspaces $U,W\leq V$.
\begin{lemma}\label{lem:wittPrep1}
There are subspaces $U'\geq U$ and $W'\geq W$ with $U',W'$ nondegenerate such that $\alpha$ can be extended to an isometry $\tilde{\alpha}\colon U'\to W'$.
\end{lemma}
\begin{proof}
We show this claim using induction on $n=\dim (U\cap U^\bot)$.

If $n=0$ then $U$ itself is nondegenerate and we are done.

If $n>0$ then let $r\in (U\cap U^\bot)\setminus\{0\}$ and $\tilde U\leq V$ such that $\langle r\rangle\oplus \tilde U=U$. By Theorem \ref{thm:hypDecomp} there is a hyperbolic plane $H\leq V$ containing $r$ such that $\tilde U\bot H$. 
Similarly, there is a hyperbolic plane $H'\leq V$ containing $r':=\alpha(r)$ such that $H'\bot\tilde{W}$ with $\tilde{W}=\alpha(\tilde{U})$. Let $r,s$ and $r',s'$ be a basis of $H$ and $H'$, respectively. Note that $\omega(r,s)\not=0$ and $r\bot U$ imply $s\notin U$. We can assume w.l.o.g. that $\omega(r,s)=\omega(r',s')$. Then we can extend $\alpha$ to $\tilde{\alpha}\colon \langle U,s\rangle\to \langle W,s'\rangle$ by $\tilde\alpha (s)= s'$. Note that $\langle U,s\rangle=\langle \tilde{U},H\rangle$. Since $\tilde{U}\bot H$ we have
\begin{align*}
\dim(\langle \tilde{U},H\rangle \cap \langle \tilde{U},H\rangle^\bot)&=\dim(\tilde{U}\cap \tilde{U}^\bot)<\dim(U\cap U^\bot).\\
%\tilde{\alpha}\colon \langle \tilde{U},r,s\rangle\to \langle \tilde W,r',s'\rangle
\end{align*}
Hence we can apply the induction hypothesis to $\tilde{\alpha}$.
\end{proof}

\begin{lemma}\label{lem:wittPrep2}
If $U$ is nondegenerate. Then $\alpha$ can be extended to an isometry $\tilde{\alpha}\colon V\to V$.
\end{lemma}
\begin{proof}
Since $U$ and $W$ are nondegenerate we can apply Lemma \ref{lem:Ubotbot} and obtain \[V=U\oplus U^\bot=W\oplus W^\bot.\]
By Lemma \ref{lem:dimUBot} we have $\dim U^\bot=\dim W^\bot$. Hence, by Corollary \ref{cor:structureSympl}, there is an isometry $\beta\colon U^\bot\to W^\bot$. Finally, $\alpha\oplus \beta\colon V\to V$ is an isometry extending $\alpha$.
\end{proof}

With this preparation we can now come to the main result.
\begin{corollary}[Witt's Lemma]\label{lem:witt}
The map $\alpha$ can be extended to an isometry $\tilde{\alpha}\colon V\to V$.
\end{corollary}
\begin{proof}
Using Lemma \ref{lem:wittPrep1} extend $\alpha$ to $\tilde{\alpha}\colon U'\to W'$ for some $U'\geq U$, $W'\geq W$ nondegenerate. Now apply Lemma \ref{lem:wittPrep2} to extend $\tilde{\alpha}$ to $\tilde{\tilde{\alpha}}\colon V\to V$.
\end{proof}

Now that we understand symplectic spaces and can extend isometries we are well equipped for the next section, where will show that $\clim \Sp_{2^n}(q)$ is also extremely amenable.


\section{Limits of other Matrix group families are Levy groups too}\label{sec:mySec}
%When studying matrices it is often useful to look at the corresponding linear maps of a suitable vector space. In the case of orthogonal, symplectic, or unitary matrices these are linear maps from the vector space to itself preserving an orthogonal, symplectic, or unitary form respectively. Formally, the symplectic group $\Sp_n(q)$ is isomorphic to $\Aut(V,\omega)$ from Corollary \ref{cor:structureSympl} it follows that $\Sp_n(q)$ is well defined, where $V$ is an $n$-dimensional $F(q)$ vector space and $\omega$ is a symplectic form.

%As we have to handle only finite dimensional vector spaces here a lot of nice theorems hold. \dots

Our goal in this section is to show that $\clim \Sp_{2^n}(q)$ is extremely amenable. The structure of the proof is the same as in Section \ref{sec:thom} for special linear groups. We will bound the length of $\Sp_n(q)\cong\Aut(V,\omega)$ by applying Corollary \ref{cor:meaContractionGroups} to a chain of subgroups $(G_i)_i$. To bound the diameter of $G_i/G_{i-1}$ we will construct for any $g\in G_i$ an $h'\in G_i$ such that the distance between $g$ and $h'g$ is small and $h'g\in G_{i-1}$. The $h'$ will behave like the inverse of $g$ on a small subspace of $V$ and like the identity on most of the rest.    The proof can be generalized to unitary and orthogonal groups.

\begin{definition}
Let $V$ be a finite dimensional $\F_q$ vector space and $\omega$ a nondegenerate map from $V\times V$ to $\F_q$.

Then $(V,\omega)$ is an \define{orthogonal space} if $\omega$ is bilinear, $\omega(x,y)=\omega(y,x)$ for all $x,y\in V$, and if $q=2$ then $\omega(x,x)=0$ for all $x\in V$.

And $(V,\omega)$ is a \define{unitary space} if there is a $h\in\Aut(\F_q)$ with $h^2=1$ such that 
\begin{align*}
\omega(ax+y,z)&=a\omega(x,z)+\omega(y,z)\\
\omega(x,ay+z)&=h(a)\omega(x,y)+\omega(x,z)\\
\omega(x,y)&=h(\omega(y,x))
\end{align*}
for all $x,y,z\in V$ and $a\in\F_q$.

\define{Orthogonal} and \define{unitary groups} are the automorphism groups of unitary and orthogonal spaces, respectively.
\end{definition}

In the following let $(V,\omega)$ be a symplectic, unitary, or orthogonal space.
Note that $\omega$ is nondegenerate and 
\begin{align*}
\omega(x,y)=0 \text{ iff }\omega(y,x)=0\text{ for all $x,y\in V$}.
\end{align*}
Obviously, Lemmas \ref{lem:dimUBot} and \ref{lem:Ubotbot} from the previous section still hold in  unitary and orthogonal spaces. Furthermore, Witt's Lemma also holds in unitary and orthogonal spaces, for a proof see \cite{Witt}
\begin{theorem}[Witt's Lemma]\label{thm:Witt}
Let $(V,\omega)$ be a symplectic, unitary, or orthogonal space and $\alpha\colon U\to W$ be an isometry between subspaces $U,W\leq V$.
Then $\alpha$ can be extended to an isometry $\tilde{\alpha}\colon V\to V$.
\end{theorem}

The next lemma is necessary to construct the chain of subgroups, in the case of symplectic spaces it is a trivial consequence of Theorem \ref{thm:hypDecomp}.

%Let $V$ be an $n$ dimensional vector space.%and e_1,...,e_n\in V a basis
%\begin{lemma}\label{lem:complementExists}
%For all $U\leq V$ there is an $U'\leq V$ such that $U\oplus U'=V$.
%\end{lemma}


%Let $\omega$ be a bilinear form on $V$.

%\begin{lemma}\label{lem:dimComplement}
%Let $U\leq V$. Then $\dim U^\bot= \dim V-\dim U$.
%\end{lemma}
%\begin{lemma}\label{lem:doubleComplement}
%Let $U\leq V$. Then $U^{\bot^\bot}=U$.
%\end{lemma}


\begin{lemma}\label{lem:decompositionComplement}
Then there exists a $U\leq V$ with $\dim U\leq 2$ such that $V=U\oplus U^\bot$.
\end{lemma}
\begin{proof}
Let $r\in V\setminus\{0\}$. By Lemma \ref{lem:dimUBot} $\dim r^\bot= n-1$. %Since $\omega$ is non degenerate $\omega(e,.)\not=0$ and therefore $e^\bot\not=V$. Hence $\dim e^\bot= n-1$.

If $r\notin r^\bot$, then $V=\langle r\rangle\oplus r^\bot$ and $\langle r\rangle$ is the desired $U$.

If $r\in r^\bot$, then let $H\leq r^\bot$ such that $\langle r\rangle\oplus H=r^\bot$. Now procces as in the proof of Theorem  \ref{thm:hypDecomp} to show that $H^\bot$ is a suitable $U$.


%extend $e$ to a basis $e,b_2,\dots,b_{n-1}$ of $e^\bot$ and consider the 2-dimensional subspace $U:=\langle b_2,\dots,b_{n-1}\rangle^\bot$. Now we have to show that \[U\cap U^\bot=0.\]
%Take $v$ from the intersection. By Lemma \ref{lem:Ubotbot} $U^\bot=\langle b_2,\dots,b_{n-1}\rangle$ and $v\bot b_i$ for all $i\in\{2,\dots,n-1\}$. Since $\langle b_2,\dots,b_{n-1}\rangle\leq e^\bot$ we also have $v\bot e$. Hence $v\in e^{\bot^\bot}=\langle e\rangle$ and $v=\lambda e$. Now $e\notin \langle b_2,\dots,b_{n-1}\rangle$ implies $v=0$. Henceforth $V=U\oplus U^\bot$.
\end{proof}

The following lemma shows that isometries interact nicely with the complement.

\begin{lemma}\label{lem:isomStaysInCompl}
Let $U\leq V$ and $\alpha\colon V\to V$ be an isometry such that $\alpha(U)=U$. Then $\alpha(U^\bot)= U^\bot$.
\end{lemma}
\begin{proof}
As $\dim \alpha(U^\bot)=\dim U^\bot$ it suffices to show that $\alpha(u')\bot u$ for all $u\in U$ and $u'\in U^\bot$. Let $v\in U$ with $\alpha(v)=u$. Then
\begin{align*}
\omega(\alpha(u'),u)&=\omega(\alpha(u'),\alpha(v))\\
&=\omega(u',v)\\
&=0.
\end{align*}
This concludes the proof.
\end{proof}

The next lemma gives us a large subspace on which $h'$ can be the identity without interfering with the part where it is the inverse of $g$.

\begin{lemma}\label{lem:largeOrthogonal}
For all $W\leq V$ there is a $W'\leq W^\bot$ such that $W\cap W'=0$ and \[\dim W'\geq \dim V-2\dim W.\]
\end{lemma}
\begin{proof}
Let $W'\leq W^\bot$ such that 
\[W^\bot=(W^\bot\cap W)\oplus W'.\]
Clearly, $W\cap W'=0$ and 
%\[\dim W'=\dim W^\bot - \dim (W^\bot\cap W)\geq \dim W^\bot- \dim W.\]
\begin{align*}
\dim W'&=\dim W^\bot - \dim (W^\bot\cap W)\\
&\geq \dim W^\bot- \dim W\\
&=\dim V-\dim W-\dim W .\tag{Lemma \ref{lem:dimUBot}}
\end{align*}
This concludes the proof.
\end{proof}


%\begin{lemma}
%Let $U,W\leq V$ such that $U\bot W$ and $U\cap W=0$. Then $\langle U,W\rangle\cong U\oplus W$.
%\end{lemma}
%\begin{lemma}\label{lem:isomSum}\FlorianSagt{maybe $g\colon U_1\to U_2$ and $h\colon W_1\to W_2$ better}
%Let $g_1\colon U_1\to W_1$ and $g_2\colon U_2\to W_2$ be isometries such that $U_1\bot U_2$, $U_1\cap U_2=0$, $W_1\bot W_2$, and $W_1\cap W_2=0$.
%Then $g_1\oplus g_2\colon U_1\oplus U_2\to W_1\oplus W_2$ is also an isomtry.
%\end{lemma}
%\begin{proof}
%Obviously, $g_1\oplus g_2$ is again a bijective linear map. Consider $v_1+v_2, u_1+u_2\in U_1\oplus U_2$
%\begin{align*}
%\omega(v_1+v_2, u_1+u_2)&=\omega(v_1, u_1)+\omega(v_1, u_2)+\omega(v_2, u_1)+\omega(v_2, u_2)\\
%&=\omega(v_1, u_1)+0+0+\omega(v_2, u_2) \tag{$U_1\bot U_2$}\\
%&=\omega(g_1(v_1), g_1(u_1))+\omega(g_2(v_2), g_2(u_2))\\
%&=\omega(g_1(v_1), g_1(u_1))+\omega(g_1(v_1), g_2(u_2))\\ &\phantom{={}}+\omega(g_2(v_2), g_1(u_1))+\omega(g_2(v_2), g_2(u_2))\tag{$W_1\bot W_2$}\\
%&=\omega(g_1\oplus g_2(v_1+v_2), g_1\oplus g_2(u_1+u_2))
%\end{align*}
%Hence $g_1\oplus g_2$ preserves $\omega$.
%\end{proof}

Now we can proof the analogue of Theorem \ref{thm:SLConcentrates} from Section \ref{sec:thom}.

%Let $G$ be a symplectic, unitary, or orthogonal group. Then the \define{dimension} of $G$ is the minimum of $\dim V'$ such that there is an $\omega'$ with $G\cong\Aut(V',\omega')$. 

\begin{theorem}\label{thm:suoBoundDiam}
%Let $G$ be a symplectic, unitary, or orthogonal group of dimension $n\geq2$ equipped with the rank metric $d$, where $d(g,g')=\lambda \cdot\rank(g-g')$. Then there is a subgroup $H\leq G$ of dimension at most $n-1$ such that the diameter of $G/H$ is at most $8\lambda$.

Let $G$ be a symplectic, unitary, or orthogonal group equipped with the rank metric $d$ and of diameter $n$. Then there is a symplectic, unitary, or orthogonal subgroup $H\leq G$ with diameter at most $n-1$ such that the diameter of $G/H$ is at most $8$.
\end{theorem}
\begin{proof}
$G=\Aut(V,\omega)$ for some vector space $V$ with  bilinear form $\omega$. Use Lemma \ref{lem:decompositionComplement} to obtain $U\leq V$ such that $V=U\oplus U^\bot$ and $\dim U\leq2$. Define $H=\Aut(U^\bot,\omega)$. Our aim is to find for any $g\in G$ an $g'\in H$ such that $d(g,g')\leq8$.
The idea is to find a map $h'\in G$ that behaves like the inverse of $g$ on $gU$ and like the identity on most of the rest. Then $h'g$ is the desired $g'$.

Let $g\in G$ and define $W=\langle U,gU\rangle$. By Lemma \ref{lem:largeOrthogonal} there is a $W'$ such that $\dim W'\geq n-8$, $W'\leq W^\bot$, and $W'\cap W=0$. Consider the map
\begin{align*}
 g^{-1}|_{gU}\oplus 1_{W'}\colon gU\oplus W'\to U\oplus W'
\end{align*}
as $g^{-1}|_{gU}$ and $1_{W'}$ are isometries and $W\bot W'$ we have that the above map is also an isomtry. By Theorem \ref{thm:Witt} this isometry can be extended to an isometry $h'\colon V\to V$. Furthermore,
\begin{align*}
 d(g,h'g)&=\dim\im (g-h'g)\\
&\leq 8+\dim \im(g-h'g)|_{W'}\tag{$\dim W'\geq n-8$}\\
&= 8+\dim \im(g-g)|_{W'}\tag{$h'|_{W'}=1_{W'}$}\\
&=8.
\end{align*}
Finally, we need to show that $h'g\in H$, here the choice of $H$ using Lemma \ref{lem:decompositionComplement} comes into play. By construction of $h'$ we have that $h'g|_U=1_U$. Therefore we can apply Lemma \ref{lem:isomStaysInCompl} and get that $h'g(U^\bot)= U^\bot$. Hence $h'g\in H$ and $d(g,h'g)\leq8$.

\end{proof}

\begin{corollary}\label{cor:suoConcentrates}
Let $G=\Aut(V,\omega)$ be a symplectic, unitary, or orthogonal group equipped with the normalized rank metric $d$ and the normalized counting measure $\mu$, where $V$ is $n$ dimensional. Then the length of $G$ is at most $8n^{-\frac{1}{2}}$ and for all $\varepsilon>0$
\[\alpha_{(G,d,\mu)}(\varepsilon)\leq 2\exp\left(-\frac{\varepsilon^2n}{16\cdot64}\right).\]
\end{corollary}
\begin{proof}
Applying Theorem \ref{thm:suoBoundDiam} multiple times gives us a sequence of subgroups $\{e\}=G_0\leq \dots\leq G_m=G$ such that $m\leq n$ and the diameter of $G_i/G_{i-1}$ is at most $\frac{8}{n}$. Now we can use Corollary \ref{cor:meaContractionGroups} to obtain the desired upper bound.
\end{proof}


Now we can prove the main result of this thesis.
\begin{corollary}
Let $(V_0,\omega_0)\subset (V_1,\omega_1)\subset \dots$ be a sequence of $\F_q$ vector spaces such that $(V_n,\omega_n)$ is a symplectic, unitary, or orthogonal space of dimension $2^n$ and $\omega_{n+1}|_{V_n}=\omega_n$ for all $n\in\N$. Let $G_{n}=\Aut(V_n,\omega_n)$ equipped with the normalized rank metric $d_n$ and the normalized counting measure $\mu_n$. 
Then \[\lim_{n\to\infty}\alpha_{(G_n,d_n,\mu_n)}(\varepsilon)=0\] for all $\varepsilon>0$ and $\clim G_n$ is extremely amenable.
\end{corollary}
\begin{proof}
Immediate from Corollary \ref{cor:suoConcentrates} and Theorem \ref{thm:LevyImpliesExAm}.
\end{proof}

\section{Application}\label{sec:ramsey}
In this section we will use the upper bound obtained for the length of symplectic, unitary, and orthogonal groups to deduce a Ramsey theoretic result. %The proof is the same as in \cite{thom} we just replaced $\SL_n(q)$ by a symplectic, unitary, or orthogonal group.
As in Section \ref{sec:mySec} the results from this section are already shown in \cite{thom} for special linear groups. 

The first lemma is very similar to Theorem \ref{thm:measureConcetration}.
\begin{lemma}[Lemma 2.7 in \cite{thom}]\label{lem:measureOfNeigh}
Let $(X,d,\mu)$ be a finite mm-space with length $l$. Then for $\varepsilon>0$ and $A\subseteq X$ with $\mu(A)>2\exp\left(-\frac{\varepsilon^2}{16l^2}\right)$ we have
\[\mu(N_\varepsilon(A))\geq 1-2\exp\left(-\frac{\varepsilon^2}{16l^2}\right).\]
\end{lemma}


A covering $\mathcal{U}$ of a metric space $(X,d)$ is an \define{$\varepsilon$-covering} if for every $x\in X$ the $\varepsilon$-neighborhood of $x$ is contained in some $U\in\mathcal{U}$.

\begin{theorem}
Let $\varepsilon>0$, $k,m\in\N$. Define $N:=16\cdot64\varepsilon^{-2}\cdot\max\{\ln(2k),\ln(2m)\}$ and let $G=\Aut(V,\omega)$, where $(V,\omega)$ is a symplectic, unitary, or orthogonal space of dimension $n>N$, with an $\varepsilon$-cover $\mathcal{U}$ of cardinality at most $m$. Then there is a $U\in\mathcal{U}$ such that for all $F\subseteq G$ satisfying $|F|\leq k$ there is a $g\in G$ with $gF\subseteq U$.
\end{theorem}
Intuitively the theorem says that whenever we color $G$ with $m$ colors, where a single element can have multiple colors, such that all elements of $\varepsilon$-balls have at least one color in common, then there is one color $c$ such that for every $F$ with at most $k$ elements there is a $g$ where the elements of $gF$ all have the color $c$.
\begin{proof}
\def\core{\operatorname{Core}}
Look at $G$ as the usual mm-space with normalized rank metric and normalized counting measure.
Let $l$ be the length of $G$, observe that, by Corollary \ref{cor:suoConcentrates}, $l\leq 8n^{-\frac{1}{2}}$.
For $U\in \mathcal{U}$ define $\core(U):=\{x\in U\mid N_\varepsilon(x)\subseteq U\}$. Since $\mathcal{U}$ is an $\varepsilon$-covering we have $\bigcup_{U\in\mathcal{U}}\core(U)=G$. Therefore there is a $U\in\mathcal{U}$ such that $\mu(\core(U))\geq\frac{1}{m}$. As $n>16\cdot64\varepsilon^{-2}\cdot\ln(2m)$ we have
\[\frac{1}{m}>2\exp\left(-\frac{\varepsilon^2n}{16\cdot64}\right)\geq 2\exp\left(-\frac{\varepsilon^2}{ 16l^2}\right).\]
Now we can apply Lemma \ref{lem:measureOfNeigh} to $\core(U)$ and obtain
\[\mu(U)\geq\mu(N_\varepsilon(\core(U)))\geq 1-2\exp\left(-\frac{\varepsilon^2}{ 16l^2}\right)\geq1-2\exp\left(-\frac{\varepsilon^2n}{16\cdot64}\right). \]

Let $F\subseteq G$ with $|F|\leq k$. Note that 
\[\{g\in G\mid gF\subseteq U\}=\bigcap_{h\in F}\{g\in G\mid gh\in U\}=\bigcap_{h\in F}Uh^{-1}.\]
Therefore, $\mu(\{g\in G\mid gF\subseteq U\})\geq 1-k\cdot2\exp\left(-\frac{\varepsilon^2n}{16\cdot64}\right)$. By assumption $n>16\cdot64\varepsilon^{-2}\cdot\ln(2k)$, hence $\mu(\{g\in G\mid gF\subseteq U\})>0$ and there is a suitable $g$. 
\end{proof}


%\section{Conclusion}