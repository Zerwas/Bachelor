\section{Witts Lemma}

\section{Limits of other Matrix group families are Levy groups too}
When studying matrices it is often useful to look at the corresponding linear maps of a suitable vector space. In the case of orthogonal, symplectic, or unitary matrices these are linear maps from the vector space to itself preserving an orthogonal, symplectic, or unitary form respectively. Formally, the symplectic group $\Sp_n(q)$ is isomorphic to $\Aut(V,\omega)$, where $V$ is an $n$-dimensional $F(q)$ vector space and $\omega$ is a symplectic form.

As we have to handle only finite dimensional vector spaces here a lot of nice theorems hold. \dots

Let $V$ be an $n$ dimensional vector space.%and e_1,...,e_n\in V a basis
\begin{lemma}\label{lem:complementExists}
For all $U\leq V$ there is an $U'\leq V$ such that $U\oplus U'=V$.
\end{lemma}


Let $\omega$ be a bilinear form on $V$.

\begin{lemma}\label{lem:dimComplement}
Let $U\leq V$. Then $\dim U^\bot= \dim V-\dim U$.
\end{lemma}
\begin{lemma}\label{lem:doubleComplement}
Let $U\leq V$. Then $U^{\bot^\bot}=U$.
\end{lemma}

\begin{lemma}\label{lem:decompositionComplement}
There exists a $U\leq V$ with $\dim U\leq 2$ such that $V=U\oplus U^\bot$.
\end{lemma}
\begin{proof}
Let $e\in V\setminus\{0\}$. By Lemma \ref{lem:dimComplement} $\dim e^\bot= n-1$. %Since $\omega$ is non degenerate $\omega(e,.)\not=0$ and therefore $e^\bot\not=V$. Hence $\dim e^\bot= n-1$.

If $e\notin e^\bot$, then $V=\langle e\rangle\oplus e^\bot$ and $\langle e\rangle$ is the desired $U$.

If $e\in e^\bot$, then extend $e$ to a basis $e,b_2,\dots,b_{n-1}$ of $e^\bot$ and consider the 2-dimensional subspace $U:=\langle b_2,\dots,b_{n-1}\rangle^\bot$. Now we have to show that \[U\cap U^\bot=0.\]
Take $v$ from the intersection. By Lemma \ref{lem:doubleComplement} $U^\bot=\langle b_2,\dots,b_{n-1}\rangle$ and $v\bot b_i$ for all $i\in\{2,\dots,n-1\}$. Since $\langle b_2,\dots,b_{n-1}\rangle\leq e^\bot$ we also have $v\bot e$. Hence $v\in e^{\bot^\bot}=\langle e\rangle$ and $v=\lambda e$. Now $e\notin \langle b_2,\dots,b_{n-1}\rangle$ implies $v=0$. Henceforth $V=U\oplus U^\bot$.
\end{proof}


\begin{lemma}\label{lem:isomStaysInCompl}
Let $U\leq V$ and $f\colon V\to V$ be an isometry such that $f|_U=1_U$. Then $f(U^\bot)= U^\bot$.
\end{lemma}
\begin{proof}
As $\dim f(U^\bot)=\dim U^\bot$ it suffices to show that $f(u')\bot u$ for all $u\in U$ and $u'\in U^\bot$.
\begin{align*}
\omega(f(u'),u)&=\omega(f(u'),f(u))\\
&=\omega(u',u)\\
&=0
\end{align*}
This concludes the proof.
\end{proof}

\begin{lemma}\label{lem:largeOrthogonal}
For all $W\leq V$ there is a $W'\leq W^\bot$ such that $W\cap W'=0$ and \[\dim W'\geq \dim V-2\dim W.\]
\end{lemma}
\begin{proof}
By Lemma \ref{lem:complementExists} there is a $W'$ such that 
\[W^\bot=(W^\bot\cap W)\oplus W'.\]
Clearly, $W\cap W'=0$ and 
\[\dim W'=\dim W^\bot - \dim (W^\bot\cap W)\geq \dim W^\bot- \dim W.\]
Whats left is to show that $\dim W^\bot\geq \dim V-\dim W$. Let $b_1,\dots, b_{\dim W}$ be a basis of $W$. Then $W^\bot$ is equal to the kernel of the linear map 
\begin{align*}
V\to F_q^{\dim W}&& v\mapsto
\begin{pmatrix}
\omega(b_1,v)\\
\vdots\\
\omega(b_{\dim W},v)
\end{pmatrix}.
\end{align*}
Now the statement follows from the rank-nullity theorem.
\end{proof}


\begin{lemma}
Let $U,W\leq V$ such that $U\bot W$ and $U\cap W=0$. Then $\langle U,W\rangle\cong U\oplus W$.
\end{lemma}
\begin{lemma}\label{lem:isomSum}\FlorianSagt{maybe $g\colon U_1\to U_2$ and $h\colon W_1\to W_2$ better}
Let $g_1\colon U_1\to W_1$ and $g_2\colon U_2\to W_2$ be isometries such that $U_1\bot U_2$, $U_1\cap U_2=0$, $W_1\bot W_2$, and $W_1\cap W_2=0$.
Then $g_1\oplus g_2\colon U_1\oplus U_2\to W_1\oplus W_2$ is also an isomtry.
\end{lemma}
\begin{proof}
Obviously, $g_1\oplus g_2$ is again a bijective linear map. Consider $v_1+v_2, u_1+u_2\in U_1\oplus U_2$
\begin{align*}
\omega(v_1+v_2, u_1+u_2)&=\omega(v_1, u_1)+\omega(v_1, u_2)+\omega(v_2, u_1)+\omega(v_2, u_2)\\
&=\omega(v_1, u_1)+0+0+\omega(v_2, u_2) \tag{$U_1\bot U_2$}\\
&=\omega(g_1(v_1), g_1(u_1))+\omega(g_2(v_2), g_2(u_2))\\
&=\omega(g_1(v_1), g_1(u_1))+\omega(g_1(v_1), g_2(u_2))\\ &\phantom{={}}+\omega(g_2(v_2), g_1(u_1))+\omega(g_2(v_2), g_2(u_2))\tag{$W_1\bot W_2$}\\
&=\omega(g_1\oplus g_2(v_1+v_2), g_1\oplus g_2(u_1+u_2))
\end{align*}
Hence $g_1\oplus g_2$ preserves $\omega$.
\end{proof}

[other useful theorems]


\begin{theorem}[Witt]
Let $V$ be an orthogonal, symplectic, or unitary space. Let $U$ and $W$ be subspaces of $V$ and suppose $\alpha\colon U\to W$ is an isometry. Then $\alpha$ extends to an isometry of $V$.
\end{theorem}

\begin{lemma}
Let $G$ be an orthogonal, symplectic, or unitary group. \dots
\end{lemma}
\begin{proof}
$G=\Aut(V,\omega)$ for some vector space $V$ with  bilinear form $\omega$. Use Lemma \ref{lem:decompositionComplement} to obtain $U\leq V$ such that $V=U\oplus U^\bot$ and $\dim U\leq2$. Define $H=\Aut(U^\bot,\omega)$. Our aim is to find for any $g\in G$ an $g'\in H$ such that $d(g,g')\leq\frac{8}{n}$.
The idea is to find a map $h\in H$ that behaves like the inverse of $g$ on $gU$ and like the identity on most of the rest. Then $hg$ is the desired $g'$.

Let $g\in G$ and define $W=\langle U,gU\rangle$. By Lemma \ref{lem:largeOrthogonal} there is a $W'$ such that $\dim W'\geq n-8$, $W'\leq W^\bot$, and $W'\cap W=0$. Consider the map
\begin{align*}
 g^{-1}|_{gU}\oplus 1_{W'}\colon gU\oplus W'\to U\oplus W'
\end{align*}
as $g^{-1}|_{gU}$ and $1_{W'}$ are isometries and $W\bot W'$ Lemma \ref{lem:isomSum} implies that the above map is also an isomtry. By Witt's lemma this isometry can be extended to an isometry $h\colon V\to V$. 
\begin{align*}
n\cdot d(g,hg)&=\dim\im (g-hg)\\
&\leq 8+\dim \im(g-hg)|_{W'}\tag{$\dim W'\geq n-8$}\\
&= 8+\dim \im(g-g)|_{W'}\tag{$h|_{W'}=1_{W'}$}\\
&=8
\end{align*}
Finally, we need to show that $hg\in H$, here the choice of $H$ using Lemma \ref{lem:decompositionComplement} comes into play. By construction of $h$ we have that $hg|_U=1_U$. Therefore we can apply Lemma \ref{lem:isomStaysInCompl} and get that $hg(U^\bot)= U^\bot$. Hence $hg\in H$ and $d(g,hg)\leq\frac{8}{n}$.

\end{proof}