\section{Martingale Technique}
\begin{definition}
A \define{martingale} is a family $(f_i,\mathcal{F}_i)_{i\in \{0,\dots,n\}}$ such that
\begin{itemize}
\item $f_i$ is integrable for all $i\in \{0,\dots,n\}$,
\item $f_i$ is $\mathcal{F}_i$ measurable for all $i\in \{0,\dots,n\}$, and
\item $f_i=\mathbb{E}[f_{i+1}|\mathcal{F}_i]$ for all $i\in \{0,\dots,n-1\}$.
\end{itemize}
\end{definition}

\begin{lemma} For all $x\in\R$
\[e^{x}\leq x+e^{x^2}.\]
\end{lemma}

\begin{lemma}[Azema's inequality]
\[\mu(\{x\in X\mid |f(x)-\mathbb{E}(f)|\geq c\})\leq 2\exp\left(-\frac{c^2}{4\sum_{i=1}^{n}\|d_i\|^2_\infty}\right)\]
\[\mu(\{|f-\mathbb{E}(f)|\geq c\})\leq 2\exp\left(-\frac{c^2}{4\sum_{i=1}^{n}\|d_i\|^2_\infty}\right)\]
\end{lemma}

\begin{definition}\label{def:length}
Let $(X,d,\mu)$ be an mm-space.
\end{definition}

\begin{theorem}
If an mm-space $(X,d,\mu)$ has length $l$, then the concentration function of $X$ satisfies
\[\alpha_X(\varepsilon)\leq 2\exp\left(-\frac{\varepsilon^2}{16l^2}\right).\]
\end{theorem}


\begin{theorem}
Let $G$ be a compact group with a bi-invariant metric $d$, and let
\[\{e\}=G_0<G_1<\dots <G_n=G\]
be a chain of subgroups. Denote the diameter of $G_i/G_{i_-1}$ with respect to the factor metric by $a_i$. Then the concentration function of the mm-space $(G,d,\mu)$, where $\mu$ is the normalized Haar measure, satisfies
\[\alpha_X(\varepsilon)\leq 2\exp\left(-\frac{\varepsilon^2}{16\sum_{i=1}^{n}a_i^2}\right).\]
\end{theorem}
%rank from 2 to 3 in proof

\begin{theorem}
The normalized counting measure on the groups $\SL_{2^n}(q)$ concentrates with respect to the rank-metric, i.e. for all $r>0$
\[\lim_{n\to\infty} \alpha_{\SL_{2^n}}(r)=0 .\]
\end{theorem}

\begin{lemma}
Let $(X,d,\mu)$ be an mm-space with diameter $d$ and \FlorianSagt{could be that this only holds for finite $X$ as conditions in definition of length are just almost surely}
\[\Omega_0=\{X\}\prec\dots\prec\Omega_n=\{\{x\}\mid x\in X\}\]
with $a_1,\dots,a_n$ as in Definition \ref{def:length}. Then 
\[\sum_{i=1}^{n}a_i\geq d.\]
\end{lemma}
\begin{proof}
Let $x,y\in X$, with $x\neq y$, we show $d(x,y)\leq\sum_{i=1}^{n}a_i$. Let $i_0$ be the smallest number such that $[x]_{i_0}\neq [y]_{i_0}$. Since $[x]_0=X=[y]_0$ we know that $i_0$ is at least 1. Therefore $[x]_{i_0-1}=[y]_{i_0-1}$ and there is an isomorphism $\varphi_{i_0}\colon[x]_{i_0}\to[y]_{i_0}$ such that $d(\varphi_{i_0}(x),y)\leq a_{i_0}$.\FlorianSagt{here is the a.s. problem} Let $x_{i_0}=\varphi_{i_0}(x)$, then 
\[d(x,y)\leq d(x,x_{i_0})+d(x_{i_0},y).\]
If $x_{i_0}=y$, then we are done. Otherwise let $i_1$ be the smallest number such that $[x_{i_0}]_{i_1}\neq [y]_{i_1}$. Then let $\varphi_{i_1}\colon[x_{i_0}]_{i_1}\to[y]_{i_1}$ be an isomorphism such that $d(\varphi_{i_1}(x_{i_0}),y)\leq a_{i_1}$. Define $x_{i_1}=\varphi_{i_1}(x_{i_0})$. Proceeding in this fashion yields elements $x_{i_0},\dots,x_{i_k}$ such that $x_{i_k}=y$ and
\[d(x,y)\leq d(x,x_{i_0})+d(x_{i_0},x_{i_1})+\dots+d(x_{i_{k-1}},x_{i_k})\leq a_{i_0}+\dots +a_{i_k}\leq\sum_{i=1}^{n}a_i.\] 
\end{proof}

\begin{lemma}
Let $(X,d,\mu)$ be an mm-space with diameter 1 and $\Delta=\min d$. Then the length of $X$ is at least $\Delta^{\frac{1}{2}}$.
\end{lemma}

\begin{definition}
The \define{symplectic group} of degree $2n$ over a field $q$, denoted by $\Sp(2n,q)$, is the subgroup of $\SL(2n,q)$ containing all matrices $A$ such that
\[A^T\Omega A=\Omega,\ \text{where}\ \Omega=\left(\begin{array}{cc}
0&E_n\\
-E_n&0
\end{array}\right).\]
\end{definition}

\begin{lemma} %d(A0,A0)
              % (CD 0I)
Let $g\colon V\to V$ be an isomorphism, $V=U\oplus U'$, and $g(U')\subseteq U'$. %it follows g(U')=U'
Then the map 
\begin{align*}
g'\colon V&\to V\\
v&\mapsto
\begin{cases}
g(v)-\pi_{U'}(g(v)) &\text{if }v\notin U'\\
v &\text{if }v\in U'
\end{cases}
\end{align*}
i.e. $g'=\pi_U\circ g-\pi_U\circ g \circ1_{U'} +1_{U'}$, is an isomorphism and $d(g,g')\leq \frac{1}{n}\cdot\dim U'$.
\end{lemma}

\begin{lemma}[what we still need (add conditions for $\omega$ if necessary)]
Let $\omega\colon V\times V\to k$ be a bilinear map, $U,U'$ subspaces of $V$, and $h\colon U\to U'$ an isomorphism that preserves $\omega$. Then $h$ can be extended to an isomorphism on $V$ which also preserves $\omega$.
\end{lemma}