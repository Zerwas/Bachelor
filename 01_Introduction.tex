%\section*{Preface}

\section{Introduction}
The objects studied in this thesis are groups of matrices over some finite field.
Given such group, we can equip it with the normalized rank metric and the normalized Haar measure to obtain a \emph{metric measure space}. Observe that, since matrix groups over finite fields are again finite, the normalized Haar measure is just the normalized counting measure.
For some sequences of matrix groups of increasing size there is a well defined limit. For example, this is the case for the special linear group of $(n\times n)$-matrices $\SL_n$.
Carderi and Thom  showed in \cite{thom} that a suitable limit of $\SL_n$ is, as a topological group, \emph{extremely amenable}. The goal of this thesis is to generalize this result to limits of other families of matrix groups, namely groups of \define{symplectic}, \define{unitary}, and \define{orthogonal matrices}.  
These matrices can be seen as \define{Isometries}, i.e. bijective linear maps from a vector space into itself preserving a symplectic, unitary, or orthogonal form.
The general strategy to prove extreme amenability for limits of families of these groups will be as follows: given such a family $(G_n)_{n\in\N}$ of finite matrix groups considered as metric measure spaces. We apply a consequence of Azema's inequality \cite{Azema} to obtain an upper bound for the \emph{measure concentration function} of $G_n$ in terms of the \define{length} of $G_n$.
As the upper bounds converge to zero we conclude that $(G_n)_{n\in\N}$ is a \emph{L\'{e}vy family}, making their limit a \emph{L\'{e}vy group}. Finally, we know from \cite{Levy} that every L\'{e}vy group is extremely amenable.

This thesis is structured as follows. In Section \ref{sec:generalDefns}, we will give a short introduction on how to view matrix groups as metric measure spaces and how to define a limit of a sequence of matrix groups. Furthermore we will introduce the notion of extreme amenability and its connection to L\'evy groups. In Section \ref{sec:azema}, we will briefly introduce the notion of \define{conditional expectation} to show Azema's inequality. Next we will introduce an important invariant of metric measure spaces, namely their \define{length}. Azema's inequality will allow us to connect the length of a metric measure space with their measure concentration function. This connection is used in Section \ref{sec:thom} to show that the limit of $\SL_n$ is extremely amenable. To generalize this result we give a proof of Witt's lemma, which says that isometries between subspaces can be extended to the whole space, in Section \ref{sec:witt}. In Section \ref{sec:mySec}, we generalize the result from Section \ref{sec:thom} to symplectic, unitary, and orthogonal groups. Finally, in Section \ref{sec:ramsey}, a Ramsey theoretic result from \cite{thom} about $\SL_n$ is generalized to symplectic, unitary, and orthogonal groups.

%Define limit of $G_n$

%Examples of matrices in the limit

%structure of thesis:

%extreme amenabiliy
%1. Azema

%2. Thoms proof (matrices as automorphisms but without form)

%3. want to generalize this so we need a form Hence extending the automorphism becomes harder so use Witts lemma

%4. generalized version of the proof

%5. application coloring theorem


%\section{Limits of Matrix Groups and Extreme Amenability}\label{sec:generalDefns}
\section{Preliminaries}\label{sec:generalDefns}
\subsection*{LAAG}
Let $q$ be a prime power and $\F_q$ be the unique $q$ element field. Denote the the general linear group over $\F_q$ by $\GL_n(q)$. We can equip $\GL_n(q)$ with the (normalized) \define{rank-metric} $d(g,h):=\frac{1}{n}\rank(g-h)$, where $r(g)$ is the rank of $g$ or equivalently the dimension of the image of $g$. 
%The group $G$, equipped with the topology induced by $d$, is a topological group.

\begin{lemma}
The metric $d$ is \define{bi-invariant}, i.e. for all $g,h,k\in \GL_n(q)$ we have 
\[d(kg,kh)=d(g,h)=d(gk,hk).\] 
\end{lemma}
\begin{proof}
Let $g,h,k\in \GL_n(q)$. Note that $k$ has full rank. Therefore
\[n\cdot d(kg,kh)=\rank(kg-kh)=\rank(k(g-h))=\rank(g-h)=n\cdot d(g,h).\]
The other equality follows similarly.
\end{proof}
%Since all matrices in $G$ have full rank, this metric is \define{bi-invariant}, i.e. for all $g,h,k\in G$ we have \[d(kg,kh)=d(g,h)=d(gk,hk).\] 


Let $G_n\leq \GL_{2^n}(q)$ be a family of subgroups, such that 
$\begin{pmatrix}
g &0\\
0&g
\end{pmatrix}\in G_{n+1}$ for all $g\in G_n$. Denote the normalized rank-metric of $G_n$ by $d_n$.
\begin{lemma}
For all $n\in \N$ the map
\[\varphi_n\colon G_n\mapsto G_{n+1}\text{, where }\varphi_n(g)=\begin{pmatrix}
g &0\\
0&g
\end{pmatrix}\]
is an isometric embedding. 
\end{lemma}
\begin{proof}
Let $n\in\N$ and $g,h\in G_n$. Then
\begin{align*}
d(g,h)&=\frac{1}{n}\rank(g-h)\\
&=\frac{1}{2n}(\rank(g-h)+\rank(g-h))\\
&=\frac{1}{2n}\rank
\begin{pmatrix}
g-h &0\\
0&g-h
\end{pmatrix}\\
&=d(
\begin{pmatrix}
g &0\\
0&g
\end{pmatrix},
\begin{pmatrix}
h &0\\
0&h
\end{pmatrix}).
\end{align*}
Clearly, $\varphi_n$ is also a homomorphism.
\end{proof}

Let $\varphi\colon\bigsqcup_{n\in\N} G_n\to\bigsqcup_{n\in\N} G_n$ with $\varphi|_{G_n}=\varphi_n$.
%Using $\varphi$ we can define the limit of $(G_n)_{n\in\N}$.
\begin{definition}
Let $\sim$ be the equivalence relation on $\bigsqcup_{n\in\N} G_n$, defined by $g\sim h$ iff there are $m,n\in\N$ such that $\varphi^n(g)=\varphi^m(h)$.
Then the \define{limit} of $(G_n)_{n\in\N}$ is defined as
\[\lim_{n\in\N} G_n:=\left(\bigsqcup_{n\in\N} G_n\right)\big/\sim.\]
\end{definition}
\begin{lemma}
The sequence of metric groups $(G_n,d_n)_{n\in\N}$ induces a group structure and a bi-invariant metric $d$ on $G=\lim G_n$. 
\end{lemma}
\begin{proof}
Note that $\psi_n\colon G_n\to G$, $\psi(g)=[g]$, is injective and \[G=\bigcup_{n\in\N}\psi_n(G_n).\]
For $[g], [h]\in G$ we can assume w.l.o.g. that $g,h\in G_n$ for some $n\in\N$. Hence we define $[g]\cdot[h]:=[gh]$ and $d([g],[h]):=d_n(g,h)$. This is well defined since the following diagram commutes for all $n\leq m$.
\begin{center}
\begin{tikzpicture}[node distance=1.8cm]
\node (G) {$G$};
\node[left of=G] (Gm) {$G_m$};
\node[below of=Gm] (Gn) {$G_n$};
\path[right hook-latex'] (Gn) edge node[below right] {$\psi_n$} (G)
          (Gn) edge node[left] {$\varphi_{n,m}$} (Gm)
          (Gm) edge node[above] {$\psi_n$} (G);
\end{tikzpicture}
\end{center}
Here $\varphi_{n,m}:=\varphi^{m-n}|_{G_n}$. Note that now the $\psi_n$ are isometric embeddings. Hence $d$ inherits all desired properties from the $d_n$'s. 
\end{proof}

Furthermore the metric and the group structure of $G$ interact nicely.
\begin{definition}
A group $G$ equipped with a topology is a \define{topological group} if 
%\begin{itemize}
the maps $G\times G\to G$, $(g,h)\mapsto gh$ and
 $G\to G$, $g\mapsto g^{-1}$ are continuous. Here we use the product topology on $G\times G$.
%\end{itemize}
\end{definition}

\begin{lemma}
Let $(G_n)_{n\in\N}$, $G=\lim G_n$, and $d$ be as before. Then $G$ with the topology induced by $d$ is a topological group.
\end{lemma}
\begin{proof}
Denote the neutral element of $G$ by $e$. First we show that the inverse is continuous. Let $\varepsilon>0$ and $g,h\in G$ with $d(g,h)<\varepsilon$. Then, by bi-invariance of $d$,
\[d(g^{-1},h^{-1})=d(e,gh^{-1})=d(h,g)<\varepsilon.\]

We use $d_\Sigma((g,h),(g',h')):=d(g,g')+d(h,h')$ as metric on $G\times G$. Let $\varepsilon>0$ and $g,g',h,h'\in G$ with $d_\Sigma((g,h),(g',h'))<\varepsilon$. Then
\[d(gh,g'h')=d(gg'^{-1},h^{-1}h')\leq d(g,g'^{-1},e)+d(e,h^{-1}h')=d(g,g')+d(h,h')<\varepsilon.\]
This yields the desired result.
\end{proof}


The group we are interested in is the metric completion of $\lim G_n$. 
\begin{lemma}
Let $G$ be a topological group with bi-invariant metric $d$. Then there a unique metric space $(\bar G,\bar d)$ containing $G$ such that $\bar G$ is complete and $G$ is dense in $\bar G$. Furthermore $\bar G$, with the group structure induced by $G$, is a topological group and $\bar d$ is still bi-invariant.
\end{lemma}
\begin{proof}
Consider the set $G_C$ of Cauchy sequences in $G$. Define $\bar G:=G_C/\sim$, where two sequences $(g_n)_{n\in\N}$ and $(h_n)_{n\in\N}$ are equivalent if $\lim d(g_n,h_n)=0$. Furthermore $\bar d([(g_n)_{n\in\N}],[(h_n)_{n\in\N}]):=\lim d(g_n,h_n)$. It is well known that $(\bar G,\bar d)$ is the unique metric completion of $(G,d)$.
%cite https://www.springer.com/de/book/9780387903125 chapter metric spaces

\[[(g_n)_{n\in\N}]\cdot[(h_n)_{n\in\N}]:=[(g_nh_n)_{n\in\N}]\]
\end{proof}
%TODO Up Next metric completion

Note that Hence we can define the inductive limit of $(G_n)_{n\in\N}$. We denote the metric completion of this limit by $\clim G_n$.%TODO definition environment
\subsection*{Topo group}
\subsection*{Levy}


%\begin{example}
%Let us consider $G:=\clim_{n\to\infty}\GL_{2^n}(q)$. The elements in $G$ are limits of elements of the form
%$\begin{pmatrix}
%g &0&\dots\\
%0&g\\
%\vdots&&\ddots
%\end{pmatrix}$
%\end{example}


\begin{lemma}\label{lem:GroupTopo}
The sequence $(G_n)_{n\in\N}$ with the normalized rank metric induces a topology on $\clim G_n$.
\end{lemma}
%\begin{proof}
%The bi-invariance of $d$ is preserved by the limit and the completion. ... 
%\end{proof}
%TODO clim does not depend on the series

Now that we have $G:=\clim G_n$ as a topological group we can ask whether it is \define{extremely amenable}, i.e. every continuous action of $G$ on a compact topological space admits a fixed point. It is hard to show this directly, but we know from \cite{Levy} that every L\'evy group is extremely amenable. Hence we will show that for suitable $(G_n)_{n\in\N}$ the limit $G$ will be a L\'evy group.

%\begin{theorem}\FlorianSagt{Not true :(}
%If $\clim G_n$ is extremely amenable then $\lim G_n$ is also extremely amenable.
%\end{theorem}
%\begin{proof}
%Every continuous action of $\lim G_n$ has a unique extension to an continuous action of $\clim G_n$. ... 
%\end{proof}

%This definition is quite hard to handle. Hence we will approach extreme amenability using L\'evy groups, as every  use the following useful theorem from \cite{levy}.
Before we can define L\'evy groups we need the following definition.
\begin{definition}
A \define{metric measure space} (mm-space) $\boldsymbol{X}$ is a triple $(X,d,\mu)$, where $d$ is a metric on the set $X$ and $\mu$ is a measure on the Borel $\sigma$-algebra induced by $d$. We will always assume that $\mu(X)=1$. 
For any set $A\subseteq X$ denote the \define{$\varepsilon$-neighborhood} of $A$, i.e. $\{x\in X\mid\exists y\in A.\  d(x,y)<\varepsilon\}$, by $N_\varepsilon(A)$.%\FlorianSagt{definition ugly, use $d_A$?}
The \define{measure concentration function} of $\boldsymbol{X}$ is defined as 
\[\alpha_{\boldsymbol{X}}(\varepsilon)=\sup\{1-\mu(N_\varepsilon(A))\mid A\subseteq X, \mu(A)\geq\frac{1}{2}\}.\]
A family of mm-spaces $\boldsymbol{X}_n$ with diameter 1 is called a \define{L\'evy family} if 
\[\alpha_{\boldsymbol{X}_n}(\varepsilon)\to 0\]
for all $\varepsilon>0$.

A topological space $\boldsymbol{X}$ is a \define{Polish space} if the topology is induced by a metric such that $\boldsymbol{X}$ as a metric space is complete and has a countable dense subset.
\end{definition}

Now we can come back to groups.
\begin{definition}
A \define{Polish group} $G$ is a topological group where the underlying topological space is a Polish space. A \define{L\'evy group} is a group $G$ equipped with a metric $d$, where
\begin{itemize}
\item $G$ with the topology induced by $d$ is a Polish group and
\item there is a sequence $(G_n)_{n\in\N}$ of compact subgroups, such that $(G_n,d|_{G_n},\mu_n)_{n\in\N}$ is a L\'evy family. Here $\mu_n$ is the normalized Haar measure of $G_n$.\FlorianSagt{$\lim G_n$ dense in $G$?}
\end{itemize}  
\end{definition}
Note that, since $G_n$ is finite, the normalized Haar measure of $G_n$ is just the normalized counting measure.
The following theorem from \cite[Theorem \textbf{4.1.3}]{Levy} %Theorem 4.1.3
gives the desired connection to extreme amenability.
\begin{theorem}\label{thm:LevyImpliesExAm}
Every L\'evy group is extremely amenable.
\end{theorem}

To apply this theorem to our setting we need the following lemma.
\begin{lemma}
Let $G_n\leq\GL_{2^n}(q)$ with the normalized rank metric $d_n$ and $G=\clim\limits_{n\to\infty} G_n$. Then $G$ equipped with the metric induced by $d_n$ is a Polish group. 
\end{lemma}
\begin{proof}
By Lemma \ref{lem:GroupTopo} $G$ is already a topological group and by definition it is also a complete metric space. Furthermore, every $G_n$ is finite. Hence the inductive limit of the $G_n$ is a countable dense subset of $G$.
\end{proof}

Whether $G$ is also a L\'evy group depends on the particular choice of $(G_n)_{n\in\N}$. 
To show that for certain sequences $G$ will be a L\'evy group, we will bound $\alpha_{G_n}(\varepsilon)$ in terms of $n$ and $\varepsilon$ and show that this bound converges to 0 as $n$ tends to infinity. 
The next section develops the methods necessary to obtain this upper bound.
%To do this we need methods which are developed in the next section.

%\[\varphi_n\colon g\mapsto\]
%is a well defined embedding