%\section*{Preface}

\section{Introduction}
If we have a finite group of matrices, then we can equip it with the normalized rank metric and the normalized Haar measure to obtain a \emph{metric measure space}.
For some sequences of matrix groups of ever larger matrices there is a well defined limit.
Carderi and Thom  showed in \cite{thom} that the closure of the limit of $\SL_n(q)$ is \emph{extremely amenable}. The goal of this thesis is to generalize this result to limits of other matrix group families, namely symplectic, unitary, and orthogonal matrices. The general strategy will be the following: given a family $(G_n)_{n\in\N}$ of (mm) matrix groups we first find an upper bound for the \emph{concentration function} of $G_n$ using a consequence of Azema's inequality \cite{Azema}. As the upper bounds converge to zero we conclude that $(G_n)_{n\in\N}$ is a \emph{L\'{e}vy family}, making the closure of their limit a \emph{L\'{e}vy group}. Finally, we know from \cite{Levy} that every L\'{e}vy group is extremely amenable.

This thesis is structured as follows. In Section \ref{sec:generalDefns} we will give a short introduction on how to see matrix groups as metric measure spaces and how to define a limit of a sequence of matrix groups. Furthermore we will introduce the notion of extreme amenability and its connection to Levy groups. In Section \ref{sec:azema} we will briefly look at conditional expectation to show Azema's inequality. This will allow us to connect the length of a metric measure space with the measure concentration function. This connection is used in Section \ref{sec:thom} to show that the closure of the limit of $\SL_n$ is extremely amenable. To generalize this result we prove Witt's lemma, which says that isometries can be extended, in Section \ref{sec:witt}. In Section \ref{sec:mySec} we generalize the result from Section \ref{sec:thom} to symplectic, unitary, and orthogonal groups. Finally, in Section \ref{sec:ramsey} a Ramsey theoretic result from \cite{thom} about $\SL_n(q)$ is generalized to symplectic, unitary, and orthogonal groups.

%Define limit of $G_n$

%Examples of matrices in the limit

%structure of thesis:

%extreme amenabiliy
%1. Azema

%2. Thoms proof (matrices as automorphisms but without form)

%3. want to generalize this so we need a form Hence extending the automorphism becomes harder so use Witts lemma

%4. generalized version of the proof

%5. application coloring theorem


\section{Limits of matrix groups and extreme amenability}\label{sec:generalDefns}
Let $\GL_n(q)$ be the general linear group over the $q$ element field $\F_q$ and let $G$ be a subgroup of $\GL_n(q)$. We can equip $G$ with the (normalized) \define{rank-metric} $d(g,h):=\frac{1}{n}\rank(g-h)$.
%The group $G$, equipped with the topology induced by $d$, is a topological group.
Since all matrices in $G$ have full rank, this metric is bi-invariant, i.e. $d(kg,kh)=d(g,h)=d(gk,hk)$ for all $g,h,k\in G$. 
Let $G_n\leq \GL_{2^n}(q)$ be a family of subgroups, such that 
$\begin{pmatrix}
g &0\\
0&g
\end{pmatrix}\in G_{n+1}$ for all $g\in G_n$. Note that the map
\[\varphi_n\colon G_n\mapsto G_{n+1}\text{, where }\varphi_n(g)=\begin{pmatrix}
g &0\\
0&g
\end{pmatrix}\]
is an isometric homomorphism for all $n\in\N$. Hence we can define the inductive limit of $(G_n)_{n\in\N}$. We denote the metric completion of this limit by $\clim\limits_{n\to\infty} G_n$.%TODO definition environment


%\begin{example}
%Let us consider $G:=\clim_{n\to\infty}\GL_{2^n}(q)$. The elements in $G$ are limits of elements of the form
%$\begin{pmatrix}
%g &0&\dots\\
%0&g\\
%\vdots&&\ddots
%\end{pmatrix}$
%\end{example}


\begin{lemma}\label{lem:GroupTopo}
The group $\clim\limits_{n\to\infty} G_n$ is a topological group.
\end{lemma}
\begin{proof}
The bi-invariance of $d$ is preserved by the limit and the completion. ... 
\end{proof}
%TODO clim does not depend on the series

Now that we have a topology on $G:=\clim\limits_{n\to\infty} G_n$ we can ask whether it is \define{extremely amenable}, i.e. every continuous action of $G$ on a compact topological space admits a fixed point. It is hard to show this directly, but we know that every L\'evy group is extremely amenable. Hence we will show that for suitable $(G_n)_{n\in\N}$ the limit $G$ will be a L\'evy group.

%This definition is quite hard to handle. Hence we will approach extreme amenability using L\'evy groups, as every  use the following useful theorem from \cite{levy}.
Before we can define L\'evy groups we need the following definition.
\begin{definition}
A \define{metric measure space} (mm-space) $\boldsymbol{X}$ is a triple $(X,d,\mu)$, where $d$ is a metric on the set $X$ and $\mu$ is a measure on the Borel $\sigma$-algebra induced by $d$. We will always assume that $\mu(X)=1$. 
For any set $A\subseteq X$ denote the $r-neighborhood$ of $A$, i.e. $\{x\in X\mid\exists y\in A.\  d(x,y)<r\}$, by $N_r(A)$.\FlorianSagt{definition ugly, use $d_A$?}
The \define{measure concentration function} of $\boldsymbol{X}$ is defined as 
\[\alpha_{\boldsymbol{X}}(r)=\sup\{1-\mu(N_r(A))\mid A\subseteq X, \mu(A)\geq\frac{1}{2}\}.\]
A family of mm-spaces $\boldsymbol{X}_n$ with diameter 1 is called a \define{L\'evy family} if 
\[\alpha_{\boldsymbol{X}_n}(r)\to 0\]
for all $r>0$.

A topological space $\boldsymbol{X}$ is a \define{Polish space} if it is homeomorphic to a complete metric space that has a countable dense subset.
\end{definition}

Now we can come back to groups.
\begin{definition}
A \define{Polish group} $G$ is a topological group where the underlying topological space is a Polish space. A \define{L\'evy group} is a group $G$ equipped with a metric $d$, where
\begin{itemize}
\item $G$ with the topology induced by $d$ is a Polish group and
\item there is a sequence $(G_n)_{n\in\N}$ of compact subgroups, such that $(G_n,d|_{G_n},\mu_n)_{n\in\N}$ is a L\'evy family. Here $\mu_n$ is the normalized Haar measure of $G_n$.\FlorianSagt{$\lim G_n$ dense in $G$?}
\end{itemize}  
\end{definition}
Note that the normalized Haar measure of $G_n$ is just the normalized counting measure.
The following theorem from \cite[Theorem \textbf{4.1.3}]{Levy} %Theorem 4.1.3
gives the desired connection to extreme amenability.
\begin{theorem}\label{thm:LevyImpliesExAm}
Every L\'evy group is extremely amenable.
\end{theorem}

To apply this theorem to our setting we need the following lemma.
\begin{lemma}
Let $G_n\leq\GL_{2^n}(q)$ and $G=\clim\limits_{n\to\infty} G_n$. Then $G$ is a Polish group. 
\end{lemma}
\begin{proof}
By Lemma \ref{lem:GroupTopo} $G$ is already a topological group and by definition it is also a complete metric space. Furthermore, every $G_n$ is finite. Hence the inductive limit of the $G_n$ is a countable dense subset of $G$.
\end{proof}

Whether $G$ is also a L\'evy group depends on the particular choice of $(G_n)_{n\in\N}$. 
%TODO not extremely amenable for G_n=GL_2^n
To show that for certain sequences $G$ will be a L\'evy group, we will bound $\alpha_{G_n}(r)$. 
The next section develops the methods necessary to obtain this upper bound.
%To do this we need methods which are developed in the next section.

%\[\varphi_n\colon g\mapsto\]
%is a well defined embedding